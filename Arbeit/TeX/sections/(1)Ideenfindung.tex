\subsection{Ideenfindung}
Immer häufiger beobachte ich, auch in kleinen Wohnorten, dass sogenannte Geschwindigkeitsanzeigeanlagen aufgebaut werden, die dem Verkehrsteilnehmer die aktuell gefahrene Geschwindigkeit anzeigen.
\missingfigure[figwidth=.8\linewidth]{Hier ein Bild von einer Radaranzeige einbetten: 1 Bild mit zB 27km/h und eines mit 33km/h -> Heubach Ortseingang} %TODO
Für Kommunen ist es hierbei wichtig, den richtigen Ort zum Aufstellen einer solchen Anzeige zu wählen, um kein Schild unnötigerweise aufzustellen. Neben der Auswahl des Ortes aufgrund der Straßenführung oder Gefahrenstellen spielt es eine große Rolle, ob das Tempolimit \dots

\subsection{Lösungsansatz \textbf{UMBENENNEN, da auch der Grund der Ausarbeitung hier genannt wird (da kostspielig usw.)???}} %TODO
Bei der Recherche zur Funktion von Geschwindigkeitsmessanlagen fällt auf, dass nur Radar- bzw. Lasertechnik, hauptsächlich für mobile Geräte, oder im Boden eingelassene Kontaktschleifen bei stationären Anlagen zur Geschwindigkeitsmessung verwendet werden. Beide Optionen sind kostspielig, da spezielle Geräte angeschafft werden müssen. Radar- und Laserpistolen arbeiten nach dem Dopplerprinzip. Bei Laserpistolen werden in der Regel viele kurze, periodische Lichtimpulse ausgesendet, die von einem Fahrzeug zurückgeworfen werden. Aufgrund des Dopplereffekts sind die Zeitabstände der reflektierten Impulse kürzer als die der ausgesendeten Impulse. Mittels eines Vergleiches beider Periodendauern kann die Geschwindigkeit des Fahrzeugs ermittelt werden. Radarmesssysteme arbeiten ähnlich, unterscheidend ist jedoch, dass die Frequenz der zurückgeworfenen Radarwelle mit der gesendeten Frequenz überlagert wird. Die entstehende Schwebungsfrequenz gibt Aufschluss über die Fahrzeuggeschwindigkeit.

Ungenauigkeiten entstehen bei den genannten mobilen Messmethoden durch unsachgemäße Positionierung und Ausrichtung des Instruments zur Fahrbahn, weshalb ein Anfechten solcher Messungen vor Gericht möglich ist. \footcite{AnfechtenMobileMessmethoden}\\

Zur Kostensenkung soll deshalb eine Software entwickelt werden, die aufgrund von Audiodaten eine Geschwindigkeitsberechnung vorbeifahrender Fahrzeuge durchführen kann. Ziel ist es, diese Software auf Smartphones einzusetzen. Weiterer Vorteil einer Analyse auf Grundlage von Geräuschen ist, dass kein Messgerät ausgerichtet werden muss, da sich der Schall der Kfz kugelförmig, das heißt nicht gerichtet, ausbreitet. Somit kann die Bedienung erleichtert werden\todo[inline]{Letzten Satz weglassen?} %todo