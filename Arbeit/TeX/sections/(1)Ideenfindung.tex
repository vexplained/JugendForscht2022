\subsection{Ideenfindung}
Aufgrund bestehender privater Projekte hat mich meine Mathematik-Lehrerin Frau Reimer dazu angeregt, bei Jugend forscht teilzunehmen. Allerdings haben diese Projekte kein Problem abgebildet, sondern waren im wesentlichen Reproduktionen bestehender Software. Gerade zu dieser Zeit haben wir im Physik-Unterricht im Zusammenhang mit mechanischen Wellen eine Abituraufgabe (Haupttermin Physik, Aufgabe I, 3.) zur Berechnung der Geschwindigkeit eines Rennwagens bearbeitet. Gegeben war lediglich das Verhältnis der Tonhöhe bei Annäherung und Entfernung des Wagens. Die Aufgabe hat sich dabei auf das Geräusch des Motors bezogen, dessen Frequenz während des Vorbeifahrens am Beobachter konstant blieb.

Da diese Art der Geschwindigkeitsberechnung für mich bisher unbekannt war, stellte ich mir die Frage, weshalb diese Methode nicht für die mobile Geschwindigkeitsmessung eingesetzt wird. Die üblichen Verfahren der mobilen Messung verwenden bisher spezielle Messgeräte, die oft sehr kostspielig sind. \cite{BushnellRadarGun}

\subsection{Lösungsansatz}
Bei der Recherche zur Funktion von Geschwindigkeitsmessanlagen fällt auf, dass nur Radar- bzw. Lasertechnik, hauptsächlich für mobile Geräte, oder im Boden eingelassene Kontaktschleifen bei stationären Anlagen zur Geschwindigkeitsmessung verwendet werden. Beide Optionen sind kostspielig, da spezielle Geräte angeschafft werden müssen. Radar- und Laserpistolen arbeiten nach dem Dopplerprinzip. Bei Laserpistolen werden in der Regel viele kurze, periodische Lichtimpulse ausgesendet, die von einem Fahrzeug zurückgeworfen werden. Aufgrund des Dopplereffekts sind die Zeitabstände der reflektierten Impulse kürzer als die der ausgesendeten Impulse. Mittels eines Vergleiches beider Periodendauern kann die Geschwindigkeit des Fahrzeugs ermittelt werden. Radarmesssysteme arbeiten ähnlich, unterscheidend ist jedoch, dass die Frequenz der zurückgeworfenen Radarwelle mit der gesendeten Frequenz überlagert wird. Die entstehende Schwebungsfrequenz gibt Aufschluss über die Fahrzeuggeschwindigkeit.

Ungenauigkeiten entstehen bei den genannten mobilen Messmethoden durch unsachgemäße Positionierung und Ausrichtung des Instruments zur Fahrbahn, weshalb ein Anfechten solcher Messungen vor Gericht möglich ist. \cite{AnfechtenMobileMessmethoden}\\

Zur Kostensenkung und Eliminierung dieser Ungenauigkeiten soll deshalb eine Software entwickelt werden, die auf Basis von Audiodaten eine Geschwindigkeitsberechnung vorbeifahrender Fahrzeuge durchführen kann. Ziel ist es, diese Software auf Smartphones einzusetzen. Weiterer Vorteil einer Analyse auf Grundlage von Geräuschen ist, dass kein Messgerät ausgerichtet werden muss, da sich der Schall der Kfz kugelförmig, das heißt nicht gerichtet, ausbreitet. Somit kann der Messablauf erleichtert werden.