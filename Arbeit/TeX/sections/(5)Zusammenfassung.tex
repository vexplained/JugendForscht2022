Eine Geschwindigkeitsmessung mit dem Smartphone durchzuführen stellen sich viele bestimmt sehr spannend vor. Aber abgesehen davon, dass man den Nachbar überführen kann, wenn dieser 10 Kilometer pro Stunde zu schnell in seine Hofeinfahrt fährt, ist das Konzept auch für Behörden nützlich. Bisherige Systeme zur mobilen Messung erfordern kostspielige Geräte, die sich insbesondere kleinere Kommunen häufig nicht leisten können. Um besonders an Durchgangsstraßen gezielt stationäre Geschwindigkeitsüberwachungsanlagen aufzubauen, ist die Messung mit Smartphones besonders praktisch, da heutzutage fast jeder ein solches Gerät bei sich trägt.

\vspace{5mm}

\noindent
Die Entwicklung einer Geschwindigkeitsmessung auf Grundlage von Audiodaten gestaltet sich als höchst komplex. Vor allem die große Anzahl an Störfaktoren und daraus resultierenden Fehlerquellen bei der Analyse erschwerten die Erarbeitung einer funktionierenden Lösung.

Ziel dieser Arbeit war es, eine physikalische Berechnung herzuleiten, die es ermöglicht, auf Basis von Audioaufnahmen von Geräten wie Smartphones eine Geschwindigkeitsberechnung vorbeifahrender Fahrzeuge durchzuführen. Anschließend sollte eine App für Smartphones entwickelt werden, die eine mobile Messung durch den Laien ermöglicht. Im Wesentlichen wurden dafür zwei Ansätze verfolgt.

Der erste Ansatz beschäftigte sich mit der Veränderung der Tonhöhe, die ein Fahrzeug von sich gibt, wenn es sich bewegt, da Schallwellen gestaucht bzw. gestreckt werden. Der sogenannte Dopplereffekt stellte sich jedoch in diesem Anwendungsfall als unbrauchbar heraus. Da die Reifen in der Regel die lauteste Geräuschquelle eines Kraftfahrzeugs sind, wurde das von den Reifen produzierte Rauschen auf Verschiebungen in der Frequenz analysiert. Es konnten jedoch keine Muster entdeckt werden, die dem Dopplereffekt entsprechen. Genauso wenig konnte das deutlich leisere Motorengeräusch für eine Analyse der Frequenzverschiebung herangezogen werden. Zwar bildet dieses einen eindeutigen Ton, aufgrund der geringen Lautstärke sind die Messergebnisse jedoch unzureichend. Aufgrund des Scheiterns einer Geschwindigkeitsberechnung per Dopplereffekt musste nach einem neuen Ansatz gesucht werden.

Das zweite Konzept beschäftigte sich mit der Auswertung der Lautstärke im Verlauf der Zeit. Anfängliche Recherchen vermittelten den Eindruck eines eindeutigen physikalischen Zusammenhangs und damit verbunden einer verhältnismäßig simplen Berechnung der Geschwindigkeit. Allerdings stellte sich die Auswertung der Rohdaten als deutlich komplexer dar, als angenommen. Durch Ungenauigkeiten in der Aufnahme gestaltete sich die Auswertung als schwierig, insbesondere da die verwendeten physikalischen Gesetze nur begrenzt und unter idealen Bedingungen nutzbar sind. Dennoch ist die Berechnung theoretisch möglich.

Trotz allem sind bisherige Ergebnisse vielversprechend, wenn auch noch nicht zuverlässig. Aus diesem Grund wird weiter nach einer Lösung der anfänglichen Fragestellung gesucht.

Vielversprechend, wenn auch höchst komplex, ist die Verwendung einer KI. Da diese nicht auf Grundlage von physikalischen Beziehungen entscheidet, sondern durch einen Vergleich mit bekannten Audioaufnahmen die Geschwindigkeit eines Kfz bestimmt, müssen Störfaktoren nicht kompensiert werden.