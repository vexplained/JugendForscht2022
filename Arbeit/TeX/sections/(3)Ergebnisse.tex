Unglücklicherweise sind die Berechnungsergebnisse sehr unterschiedlich ausgefallen. Der in \autoref{section:EntwicklungSoftware} beschriebene Algorithmus lieferte für eine Reihe an Aufnahmen von innerorts fahrenden Fahrzeugen Werte zwischen \(70 km/h\) und \(280 km/h\). Zur Analyse des Fehlers wurden mehrere Änderungen im Programmablauf vorgenommen.

Als ersten Schritt wurden sämtliche Durchschnittsberechnungen bei der Analyse der Audiodaten verworfen. Stattdessen wurden Annäherung und Entfernung des Fahrzeugs sowie linker und rechter Stereokanal separat betrachtet, wie bereits in \autoref{section:AnalyseAmplitude} beschrieben. Die erhoffte Verbesserung der Messergebnisse blieb jedoch aus: im Durchschnitt sind die deutlich zu hohen Geschwindigkeiten nur um etwa \(7 km/h\) gesunken.

Als weitere Fehlerquelle wurde die Annäherung der Hyperbelfunktion identifiziert. Obwohl der Näherungsalgorithmus mit verschiedenen Einstellungen zur Gewichtung der Werte getestet und optimiert wurde, scheiterte eine korrekte Annäherung bei falsch gewählter Dauer der auszuwertenden Audioaufnahme, insbesondere bei Aufnahmen, die Stille nach Vorbeifahren des Fahrzeugs enthalten (zu langes Messintervall). Zudem ist die Amplitudenfunktion nur begrenzt hyperbelförmig, in etwa ab ihrem Wendepunkt.

Erste Tests ergaben, dass eine Geschwindigkeitsberechnung auf Grundlage der Rohdaten anstatt der Hyperbelfunktion deutlich realistischere Ergebnisse liefert, in etwa im Bereich von \(30\) bis \(80 km/h\). Aus Zeitgründen steht die programmatische Ausarbeitung und genauere Analyse dieses Ansatzes jedoch noch aus.
