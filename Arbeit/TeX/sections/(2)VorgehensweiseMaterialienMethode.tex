\subsection{Erste Datensammlung und manuelle Aufbereitung}
Für eine möglichst gute Datengrundlage wurde an einer geraden Straße Aufnahmen von insgesamt 21 vorbeifahrenden Kfz gemacht, sowohl von dicht aufeinanderfolgenden, als auch einzelnen Fahrzeugen. Als Aufnahmegerät wurde ein Smartphone mit integrierter Rekorder-App verwendet. Die zusammenhängende Aufnahme aller Fahrzeuge wurde anschließend von Hand in einzelne Abschnitte unterteilt und als WAV-Audiodateien gespeichert. \draftred{Dieses unkomprimierte Format wurde gewählt, um die Implementierung der Datenanalyse zu erleichtern.}\todo[inline]{Andere Begründung?} %todo

\subsection{Analyse der Audiodaten via Dopplereffekt}
Da im Phy\-sik-Unter\-richt eine Abituraufgabe zur Geschwindigkeitsbestimmung eines Rennwagens mittels Differenz der Frequenz bei Annäherung und Entfernung behandelt wurde, ist das der erste verfolgte Ansatz. Es erscheint zudem einfach, die Geschwindigkeit akkurat zu ermitteln, da selbst ein Mensch eindeutige Frequenzveränderungen hören kann, beispielsweise bei einem vorbeifahrenden Krankenwagen mit Martinshorn. Allerdings muss bei normalen Kfz das Reifengeräusch anstelle des Martinshorns verwendet werden, da dieses mit Abstand die lauteste Geräuschquelle des Straßenverkehrs ist.

Wenn der Abstand des vorbeifahrenden Fahrzeugs zum Beobachter vernachlässigt und von konstanter Bewegungsgeschwindigkeit ausgegangen wird, können folgende Formeln zur Berechnung der Geschwindigkeit verwendet werden:

\[
    f_{1} = f_{0} * \frac{c}{c - v}
    \quad\text{und}\quad
    f_{2} = f_{0} * \frac{c}{c + v}
\]

Dabei ist \(f_{1}\) die Frequenz bei Annäherung und \(f_{2}\) die vom Beobachter registrierte Frequenz bei Entfernung des Fahrzeugs. Durch Messung beider Frequenzen kann das Frequenzverhältnis \(k = \frac{f_{1}}{f_{2}}\) berechnet und nach \(v\) umgestellt werden:

\begin{equation*}
    \begin{split}
        k & = \frac{f_{0} * \frac{c}{c - v}}{f_{0} * \frac{c}{c + v}} \\
        k & = \frac{c + v}{c - v} \\
        & \Leftrightarrow \\
        v & = \frac{k - 1}{k + 1} * c
    \end{split}
\end{equation*}

Für einen ersten Überblick wurden die Audio-Abschnitte in einen Spektrumanalysator geladen. Die Ergebnisse der visuellen Analyse sind in \autoref{img:spectrumanalyzer} dargestellt.

\begin{figure}[h]
    \begin{subfigure}{.5\textwidth}
        \centering
        \includegraphics[width=.8\linewidth]{Frequenzen}
        \caption{Spektrogramm}
    \end{subfigure}
    \begin{subfigure}{.5\textwidth}
        \centering
        \includegraphics[width=.8\linewidth]{Tonhöhe(EAC)}
        \caption{Tonhöhe (EAC)}
        \label{img:spectrum_b}
    \end{subfigure}
    \caption{Ergebnisse Spektrumanalysator}
    \label{img:spectrumanalyzer}
\end{figure}

\autoref{img:spectrum_b} wurde nachbearbeitet. Die pinkfarbene Linie zeigt den Verlauf der Tonhöhe über Zeit und kann als Frequenzgraph interpretiert werden. Ab der tiefsten Stelle des Graphen ist das vorbeifahrende Kfz am nächsten zum Mikrofon.

\begin{figure}[h]
    \centering
    \import{rsc/plots/f-t-plot/}{frequenz-zeit-plot.tex}
    \caption{Beispielhafter Frequenzverlauf bei vorbeifahrendem Fahrzeug}
    \label{fig:frequencyplot}
\end{figure}

Beim Vergleich mit einem theoretisch berechneten Frequenzgraph (\autoref{fig:frequencyplot}) fällt auf, dass die aufgenommene Frequenz der Reifengeräusche vor Vorbeifahren (in der Beispielabbildung bei \(t = 10 s\)) nicht höher ist als nach dem Vorbeifahren, sondern bei niedrigerem Abstand geringer ist. Es konnte keine wissenschaftliche Erarbeitung dieses Phänomens gefunden werden, am wahrscheinlichsten ist jedoch eine Reflexion der akustischen Wellen am Boden, die mit den direkt zum Mikrofon laufenden Wellen interferieren und somit hohe Frequenzen auslöschen. Aufgrund dieser unklaren Messergebnisse kann dieser Ansatz jedoch nicht weiterverfolgt werden.

\subsection{Analyse der Audiodaten via Lautstärkeänderung}