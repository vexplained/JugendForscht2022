\documentclass[12pt, a4paper]{article}
\usepackage[german]{babel}
\usepackage{import}
\usepackage{blindtext}
\usepackage{titling}
\usepackage{datetime}
\usepackage{amsmath, graphicx, pgfplots}
\usepackage[hidelinks]{hyperref}
\usepackage{fancyhdr}
\usepackage{float}
\usepackage[top=2.5cm, bottom=2.5cm, left=2.5cm, right=2.5cm]{geometry}

% Umlaut-Unterstützung
% siehe auch https://www.namsu.de/Extra/befehle/Umlaute.html
\usepackage[utf8]{inputenc} % Umlaut-Unterstützung
\usepackage[T1]{fontenc} % Umlaut-Unterstützung für Silbentrennung

%args: x1, y1, x2, y2
\newcommand{\pointdist}[4]{sqrt((#3-#1)^2+(#4-#2)^2)}
\newcommand{\modifiedpointdist}[5]{sqrt((#3-#1)^2+(#4-#2)^2)}

\pgfplotsset{compat=1.18}
% optimize pgfplots compilation: export plot, then import into document to
% prevent unnecessary recompilation
\usepgfplotslibrary{groupplots, external}
\tikzexternalize[prefix=cache/]
% import additional pgfplots/tikz libraries (for plotting)
\usepgfplotslibrary{polar, patchplots, colorbrewer}
\usetikzlibrary{calc, positioning}


\setlength{\headheight}{15pt}
\addtolength{\topmargin}{-10pt}

\title{Konzept: Ge\-schwin\-dig\-keits\-mes\-sung durch Ge\-räusch\-ana\-ly\-se}
\author{Levin Fober}
\date{\today}

\pagestyle{fancy}
\fancyhead{} % reset everything
\fancyfoot{} % reset everything
\fancyhead[R]{Konzept}
\fancyfoot[R]{\thepage}
\fancyfoot[C]{\theauthor}
\fancyfoot[L]{Erstellt: \thedate}


\begin{document}

\maketitle

\tableofcontents

\newpage

\section{Zielsetzung und Idee}
Es soll erforscht werden, wie durch die Analyse des Geräusches eines
vorbeifahrenden Fahrzeugs die Geschwindigkeit dieses Fahrzeugs ermittelt werden
kann. \\
Dazu kommt die Berechnung des Abstandes von Sensor (zum Beispiel Handymikrofon)
zum vorbeifahrenden Fahrzeug.

\section{Ansatz}
Aufgrund des Dopplereffekts kann bei bekannter Schallgeschwindigkeit mithilfe
des gemessenen Verhältnisses von Annäherungsfrequenz und Entfernungsfrequenz die
Geschwindigkeit des vorbeifahrenden Fahrzeugs berechnet werden.\\
Sofern der Abstand des Sensors zum vorbeifahrenden Fahrzeug \(d > 0\) sein
sollte, ist der Übergang von hoher zu niedriger Frequenz nicht rechteckig (im
Frequenz-Zeit-Diagramm), sondern fließend. Über die Änderungsgeschwindigkeit
kann somit der genannte Abstand ermittelt werden.

\section{Visualisierungen}

\subsection{Ausbreitung von Schall}

\begin{minipage}{0.5\textwidth}
    \begin{figure}[H]
        \import{rsc/plots/circular-wave/}{circular-wave-decay.tex}
        \caption{Gestauchte Welle}
    \end{figure}
\end{minipage} \hfill
\begin{minipage}{0.45\textwidth}
    \begin{footnotesize}
        \emph{Information: Die Grafik basiert nicht auf physikalischen Gesetzen und ist somit fachlich falsch. Sie dient lediglich der Veranschaulichung.}
    \end{footnotesize}
\end{minipage}
\vspace{1cm}

Bei Bewegung einer Schallquelle im Raum wird die Welle in Bewegungsrichtung vor
der Quelle gestaucht, dahinter gestreckt. Die Stauchung und Streckung resultiert
in einer Veränderung Frequenz für einen feststehenden Empfänger.

\subsection{Beispielhafter Frequenzverlauf}

\begin{minipage}{0.5\textwidth}
    \begin{figure}[H]
        \import{rsc/plots/f-t-plot/}{frequenz-zeit-plot.tex}
        \caption{Gestauchte Welle}
    \end{figure}
\end{minipage} \hfill
\begin{minipage}{0.45\textwidth}
    \begin{footnotesize}
        \emph{Information: Die Grafik basiert nicht auf physikalischen Gesetzen und ist somit fachlich falsch. Sie dient lediglich der Veranschaulichung.}
    \end{footnotesize}
\end{minipage}
\vspace{1cm}

\subsubsection*{Abstand \(d = 0\)}
Die Frequenzänderung geschieht ohne Übergang. Der arithmetische Mittelwert aus
hoher und tiefer Frequenz stellt die tatsächliche Frequenz dar, die die
Schallquelle aussendet.

\subsubsection*{Abstand \(d > 0\)}
Die Frequenzänderung geschieht mit Übergang. Über den Verlauf des Übergangs kann die Entfernung von Quelle und Empfänger ermittelt werden.

\end{document}
