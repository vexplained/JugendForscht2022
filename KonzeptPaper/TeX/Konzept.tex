\documentclass[12pt, a4paper]{article}
\usepackage[german]{babel}
\usepackage{import}
\usepackage{blindtext}
\usepackage{titling}
\usepackage{datetime}
\usepackage{amsmath, graphicx, hyperref, pgfplots}
\usepackage{fancyhdr}
\usepackage[top=2.5cm, bottom=2.5cm, left=2.5cm, right=2.5cm]{geometry}

% Umlaut-Unterstützung
% siehe auch https://www.namsu.de/Extra/befehle/Umlaute.html
\usepackage[utf8]{inputenc} % Umlaut-Unterstützung
\usepackage[T1]{fontenc} % Umlaut-Unterstützung für Silbentrennung

%args: x1, y1, x2, y2
\newcommand{\pointdist}[4]{sqrt((#3-#1)^2+(#4-#2)^2)}

\pgfplotsset{compat=1.18}
% optimize pgfplots compilation: export plot, then import into document to
% prevent unnecessary recompilation
\usepgfplotslibrary{groupplots, external}
\tikzexternalize[prefix=cache/]
% import additional tikz libraries (for plotting)
\usetikzlibrary{calc}


\setlength{\headheight}{15pt}
\addtolength{\topmargin}{-10pt}

\title{Konzept: Geschwindigkeitsmessung durch Geräuschanalyse}
\author{Levin Fober}
\date{\today}

\pagestyle{fancy}
\fancyhead{} % reset everything
\fancyfoot{} % reset everything
\fancyhead[R]{Konzept}
\fancyfoot[R]{\thepage}
\fancyfoot[C]{\theauthor}
\fancyfoot[L]{Erstellt: \thedate}


\begin{document}

\maketitle

\section{Zielsetzung und Idee}
Es soll erforscht werden, wie durch die Analyse des Geräusches eines
vorbeifahrenden Fahrzeugs die Geschwindigkeit dieses Fahrzeugs ermittelt werden
kann. \\
Dazu kommt die Berechnung des Abstandes von Sensor (zum Beispiel Handymikrofon)
zum vorbeifahrenden Fahrzeug.

\section{Ansatz}
Aufgrund des Dopplereffekts kann bei bekannter Schallgeschwindigkeit mithilfe
des gemessenen Verhältnisses von Annäherungsfrequenz und Entfernungsfrequenz die
Geschwindigkeit des vorbeifahrenden Fahrzeugs berechnet werden.\\
Sofern der Abstand des Sensors zum vorbeifahrenden Fahrzeug \(d > 0\) sein
sollte, ist der Übergang von hoher zu niedriger Frequenz nicht rechteckig (im
Frequenz-Zeit-Diagramm), sondern fließend. Über die Änderungsgeschwindigkeit
kann somit der genannte Abstand ermittelt werden.

\subsection{Graph der Frequenz über Zeit}
% \import{rsc/plots/f-t-plot/}{frequenz-zeit-plot.tex}
\import{rsc/plots/circular-wave/}{circular-wave-decay.tex}

\end{document}
